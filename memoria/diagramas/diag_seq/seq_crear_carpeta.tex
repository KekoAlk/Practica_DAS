\newcommand{\seqCrearCarpeta}{
	\begin{tikzpicture}
	\begin{umlseqdiag}
	\umlobject{L}
	\umlobject{Agente}
	\umlobject{Pizarra}
	\umlobject{Usuario}
	\umlobject{Estado}
	\umlobject{Elemento}
	\umlobject{Carpeta}
	\begin{umlcall}[op=establecerConexion(), type=synchron, return=establecerConexion()]{L}{Agente}
		\begin{umlcall}[op=comprobarUsuario(), type=synchron, return=comprobarUsuario()]{Agente}{Pizarra}
			\begin{umlcall}[op=comprobarUser(), type=synchron, return=comprobarUser()]{Pizarra}{Usuario}
			\end{umlcall}
		\end{umlcall}
	\end{umlcall}
	\begin{umlcall}[op=crearCarpeta(), type=synchron, return=crearCarpeta()]{L}{Agente}
		\begin{umlcall}[op=crearCarpeta(), type=synchron, return=crearCarpeta()]{Agente}{Pizarra}
			\begin{umlcall}[op=crearCarpeta(), type=synchron, return=crearCarpeta()]{Pizarra}{Estado}
				\begin{umlcall}[op=crearCarpeta(), type=synchron, return=crearCarpeta()]{Estado}{Elemento}
					\begin{umlcall}[op=<<create>>, type=synchron]{Elemento}{Carpeta}
					\end{umlcall}
				\end{umlcall}
			\end{umlcall}
		\end{umlcall}
		\begin{umlcall}[op=actualizarEstado(), type=synchron, return=actualizarEstado()]{Agente}{Pizarra}
			\begin{umlcall}[op=actualizarEstado(), type=synchron, return=actualizarEstado()]{Pizarra}{Estado}
			\end{umlcall}
		\end{umlcall}
	\end{umlcall}

	
	\end{umlseqdiag}
	\end{tikzpicture}
}