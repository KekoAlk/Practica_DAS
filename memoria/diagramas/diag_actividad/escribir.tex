% Escribir (in)
\newcommand{\escribir}{
\begin{tikzpicture}
\tikzumlset{fill state=white}
\umlstateinitial[name=initial]
\begin{umlstate}[x=0, y=-3,name=datos]{Iniciar sesión}
\end{umlstate}

\umlstatedecision[y=-5, name=permiso] 

\begin{umlstate}[x=0, y=-8,name=comprobar]{Escribir datos (Escritura)}
\end{umlstate}

\begin{umlstate}[y=-11,name=actualizar]{Actualizar pizarra}
\end{umlstate}

\begin{umlstate}[y=-14,name=notificar]{Notificar}
\end{umlstate}

\umlstatefinal[y=-16, name=final]

\umltrans{initial}{datos}
\umltrans{datos}{permiso}
\umltrans[arg1=SI]{permiso}{comprobar}
\umltrans{comprobar}{actualizar}
\umltrans{actualizar}{notificar}
\umltrans{notificar}{final}
\umlHVHtrans[arg1=NO, arm2=-4]{permiso}{notificar}

\end{tikzpicture}
}