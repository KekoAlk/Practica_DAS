% Actualizar pizarra
\newcommand{\actualizarPizarra}{
\begin{tikzpicture}

\umlstateinitial[name=initial]
\begin{umlstate}[x=0, y=-3,name=datos]{Obtener nuevos datos de la pizarra}
\end{umlstate}

\begin{umlstate}[x=0, y=-6,name=comprobar]{Conectar con la BD}
\end{umlstate}

%\umlstatedecision[y=-8, name=check] 

\begin{umlstate}[y=-9,name=sesion]{Guardar cambios en BD}
\end{umlstate}

\begin{umlstate}[y=-12,name=notificar]{Notificar}
\end{umlstate}

\umlstatefinal[y=-15, name=final]

\umltrans{initial}{datos}
\umltrans{datos}{comprobar}
\umltrans{comprobar}{sesion}
%\umltrans[arg1=Datos válidos]{check}{sesion}
\umltrans{sesion}{notificar}
\umltrans{notificar}{final}
%\umlHVHtrans[arg1=Datos incorrectos, arm2=-4,pos1=0.4]{check}{notificar}

\end{tikzpicture}
}