\chapter{Conclusiones}
\lettrine[lines=1,slope=4pt,findent=0pt]{D}{}urante todo el proceso, tanto de documentación de los requisitos como en el de análisis y el de diseño, hemos experimentado una extraña sensación, cada vez conocíamos mejor la plataforma, cada vez la entendíamos mucho mejor, hasta tal punto que prácticamente hemos sido capaces de escribir pequeñas lineas de código destinadas a la utilización de la misma.\\

Esto es debido a la gran cantidad de tiempo que le hemos dedicado y en lo que se lo hemos dedicado; centrándonos la mayor parte del tiempo en el diseño, en la creación de los distintos diagramas que sin duda alguna nos ha permitido conocer este producto.\\

Todos coincidimos en que por las fechas tan poco adecuadas, en navidad, el desarrollo de la memoria y de la práctica se ha visto truncado. Si hubiéramos tenido los mismos días para hacerla, pero en enero, seguramente nos hubiera dado tiempo a implementar completamente la plataforma e incluso a probar alguna aplicación que la implementase.\\

En cualquiera de los casos, se ha tratado de un proceso interesante cuanto menos para nosotros, hemos descubierto las propiedades del uso de arquitecturas de una manera más práctica y hemos realizado un pseudo\footnote{En realidad no hemos trabajado mediante un modelo de cascada real, más bien lo hemos simulado, puesto que algunos diagramas se hacían mientras se obtenían todavía requisitos} modelo de cascada por el cuál hemos sido participes por primera vez de un proyecto en el que se utilicen este tipo de metodologías.\\

En conclusión, podemos resumir esta práctica como única y extensa, con respecto a esto último no estamos en desacuerdo, nos parece correcto hacer trabajos de este tipo de envergaduras, pero no nos agrada que haya tenido que ser durante el período de navidad, en el que pasamos gran parte de tiempo con la familia, que nos ha impedido dedicarle tiempo a la práctica.