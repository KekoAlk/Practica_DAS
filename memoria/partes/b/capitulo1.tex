\chapter{Introducción}
\lettrine[lines=1,slope=4pt,findent=0pt]{E}{}n la segunda parte de la memoria describiremos a nivel de arquitectura y diseño una plataforma que implemente la arquitectura de pizarra. \\

En primer lugar debemos describir qué es exactamente una arquitectura de pizarra, para seguidamente poder analizar en detalle la plataforma a crear.\\

La arquitectura de pizarra puede ser presentada con la siguiente metáfora~\cite{blackboardsystem}

\begin{quote}
Imagine a group of human specialists seated next to a large blackboard. The specialists are working cooperatively to solve a problem, using the blackboard as the workspace for developing the solution.\\
Problem solving begins when the problem and initial data are written onto the blackboard. The specialists watch the blackboard, looking for an opportunity to apply their exercise to the developing solution. When a especialist find sufficient information to make a contribution, she records the contribution on the blackboard, hopefully enabling other specialists to apply their exercise. This process of adding contributions to the blackboard continues until the problem has been solved.
\end{quote}

Es decir, que una arquitectura de pizarra es un modelo arquitectónico para el diseño de software utilizada en los siguientes casos:

\begin{itemize}
	\item Problemas en los que no existe una solución analítica.
	\item Problemas en los que, aunque exista solución analítica, es inviable por la dimensión del espacio de búsqueda.
	\item Problemas extremadamente complejos en términos cognitivos.
\end{itemize}

La arquitectura de pizarra consta principalmente de dos componentes:

\begin{itemize}
	\item \textbf{Pizarra: }Instrumento de control o estructura de datos que representa el estado actual de la plataforma.
	\item \textbf{Agente: }Elementos funcionales independientes entre sí que operan sobre la pizarra.
\end{itemize}

La función de los agentes es, basándose en el contenido de la pizarra, realizar la tarea que se les asigna y escribir sobre ella sus resultados.\\

La función principal de la pizarra es coordinar la interacción de los agentes con la pizarra y la comunicación entre ellos. La pizarra irá cambiando de estado según van escribiendo en ella los distintos agentes. Éstos podrán leer lo que fue escrito por otros agentes anteriores a él. De esta forma el contenido de la pizarra será cada vez más completo, hasta alcanzar una solución final.\\

Las arquitecturas en pizarra presentan una serie de inconvenientes importantes. Estos radican principalmente en la complejidad de los problemas que se resuelven con este tipo de arquitecturas. Son los siguientes:

\begin{itemize}
	\item Una vez obtenida una solución es difícil ver de forma ordenada los pasos que llevaron a esa solución.
	\item No garantiza obtener una solución.
	\item No se puede saber a priori el tiempo necesario para realizar el problema.
\end{itemize}

\begin{figure}[!h]
\centering
\pizarra
\caption{Interacción entre los agentes y la pizarra}
\end{figure}

Una vez analizada como deberá ser la infraestrucura de pizarra, se pide diseñar una biblioteca (un framework en nuestro caso). Para ello, y siguiendo nuestro esquema general, procedemos primero a dar algunas definiciones básicas:\\

\section{¿Qué es la un framework?}
Un framework o infraestructura digital, es una estructura conceptual y tecnológica de soporte definido normalmente con artefactos o módulos de software concretos que puede servir de base para la organización y desarrollo de software.
Desde el punto de vista de su arquitectura, se basa en el modelo MVC (Controlador, Modelo, Vista), ya que debemos fragmentar nuestra programación:


\begin{itemize}
\item \textbf{Modelo:} controla las operaciones lógicas y l manejo de información.
\item \textbf{Vista:}finalmente a un miembro miembro de la familia le corresponde expresar la última forma de los datos: la interfaz gráfica que interactúa con el usuario final del programa (GUI).
\item \textbf{Controlador:} responsable de controlar el acceso a nuestra aplicación, incluyendo cualquier tipo de información que permita la interfaz.
\end{itemize}

