\chapter{Documentación}
\lettrine[lines=1,slope=4pt,findent=0pt]{U}{}na vez introducida la arquitectura de pizarra y la estructura básica que tendrá la plataforma daremos paso a una documentación detallada de la misma.\\

Para abordar la documentación de la plataforma ha sido necesario obtener previamente una lista de requisitos, para después, una vez desarrollados y clasificados convenientemente, dar lugar a la realización de los diagramas \gloss{UML} correspondientes que definen el diseño de la plataforma.\\

Ahora centrémonos en cómo hemos obtenido esos requisitos y a que conclusiones hemos llegado.

\section{Obtención de los requisitos}
Para abordar la obtención de los requisitos, y debido a que obtener los requisitos de la plataforma de la qué no teníamos una idea muy clara y concisa de su funcionalidad, hemos partido de una posible aplicación que pudiera ser implementada con nuestra librería para así poder buscar su funcionalidad, es decir, hacernos una idea clara de qué debería llevar y a partir de ahí, obtener tanto los requisitos de esta posible aplicación como separarlos y extrapolarlos a la librería.\\

La aplicación que tomamos como referencia para la extracción de requisitos permite a un profesor, así como a sus alumnos subir y modificar archivos en una pizarra externa, permitiendo así la cómoda realización de trabajos en grupo por parte de los alumnos y un seguimiento por parte del profesor.\\

Pensar de un modo más aplicado nos ha permitido extraer requisitos con más facilidad, pensando en qué cosas serían necesarias para el buen funcionamiento de la aplicación.\\

\subsection{Clasificación de requisitos}
Una vez extraídos los requisitos es necesario clarificarlos, ya que no todos hacen referencia a una misma parte de la funcionalidad, por ello se han clasificado según los siguientes criterios:

\begin{itemize}
	\item \textbf{Funcionales: }Son declaraciones de los servicios que debe proporcionar el sistema. Especifica la manera en que éste debe reaccionar a determinadas entradas. Especifica cómo debe comportarse el sistema en situaciones particulares.
	\item \textbf{No funcionales: } Restricciones de los servicios o funciones ofrecidas por el sistema (fiabilidad, tiempo de respuestas, capacidad de almacenamiento, etc.). Generalmente se aplican al sistema en su totalidad. Surgen de las necesidades del usuario debido a restricciones de presupuesto, políticas de la organización, necesidad de interoperatividad, etc. Estos a su vez se pueden dividir en:
	\begin{itemize}
		\item\textbf{Del producto:} Especifican el comportamiento del producto, por ejemplo fiabilidad o rapidez.
		\item\textbf{Organizacionales:} Derivan de políticas y procedimientos existentes en la organización del cliente y del desarrollador, por ejemplo la utilización de un lenguaje de programación en concreto.
		\item\textbf{Requisitos externos:} Se derivan de factores externos al sistema y a su proceso de desarrollo, por ejemplo requisitos de interoperatividad, la capacidad de intercambiar información de un sistema, o éticos, como el cumplimiento de normativas.
	\end{itemize}
\end{itemize}

\subsection{Pensamos en la posible aplicación}

En cuanto a la hora de pensar en qué tipo de aplicación podría ser implementada se nos ocurrieron varias ideas\cite{discusionaplicacion}, pero optamos por quedarnos con una aplicación del ámbito educativo. Concretamente, esta aplicación permitiría almacenar archivos en la pizarra o servidor y dependiendo del nivel del usuario (alumno o profesor), éste tendría habilitada unas u otras opciones, por ejemplo el profesor podría comparar los resultados de un grupo de prácticas de alumnos con los de otros para buscar copias o un alumno podría preguntar a otros grupos.\\

Tras esta pequeña descripción de la aplicación, vamos a redactar una lista con sus requisitos para poder trabajar con ellos:

\subsection{Requisitos de la aplicación}

Lista de requisitos de una posible aplicación:

\begin{itemize}
\item \textbf{Requisitos funcionales}
		\begin{itemize}
			\item La pizarra y los agentes siguen un modelo cliente-servidor.
			\item Los agentes pueden leer, escribir y modificar la pizarra.
			\item La pizarra debe permitir gestionar la BD de los distintos usuarios.
			\item Los agentes no deben poder interrelacionarse entre ellos si no es a través de la pizarra.
			\item Debe haber distintos tipos de usuarios: profesores y alumnos.
			\item La pizarra debe permitir la creación de apartados para los ejercicios, por ejemplo Tema 1.
			\item La pizarra debe guardar el progreso de cada usuario, mediante gráficos, estadísticas, etc.
			\item La pizarra debe distinguir los distintos tipos de ejercicios de los alumnos.
			\item La pizarra debe reconocer los tipos de archivo de los ejercicios.
		\end{itemize}
\item \textbf{Requisitos no funcionales}
		\begin{itemize}
			\item \textbf{De producto}
					\begin{itemize}
					\item La pizarra no debe permitir a los alumnos modificar los apuntes subidos por un profesor.
					\item La pizarra debe bloquear a los usuarios para que no puedan acceder a ejercicios de clases en las que no están inscritos.
					\item La pizarra debe almacenar una contraseña y un identificador único para cada usuario y no permitir que otros accedan.
					\item La pizarra debe permitir al profesor bloquear contenido a los alumnos.
					\end{itemize}
			\item \textbf{Organizacionales}
			\item \textbf{Externos}
		\end{itemize}
\end{itemize}

Pero, como se puede apreciar, estos requisitos hacen referencia a ciertos elementos que no pertenecen para nada a la plataforma a implementar, como pudieran ser \emph{alumno} o \emph{profesor}.

\subsection{Requisitos de la plataforma}
Para solucionar esto, se han extraído de los requisitos anteriores, la funcionalidad que va a tener nuestra plataforma y se han vuelto a redactar sus requisitos.

\begin{itemize}
\item \textbf{Requisitos funcionales}
		\begin{itemize}
			\item Los agentes pueden leer, escribir y modificar la pizarra.
			\item La pizarra debe permitir gestionar los distintos tipos de agentes.
			\item Los agentes no deben poder interrelacionarse entre ellos si no es a través de la pizarra.
			\item Debe haber distintos tipos de agentes, organizados por una jerarquía de nivel (p.e. directores, profesores, alumnos).
			\item La pizarra debe permitir la creación de carpetas donde guardar los elementos para organizarlos.
			\item La pizarra debe permitir crear nuevos usuarios.
			\item La pizarra permitirá la creación de nuevos usuarios.
			\item La pizarra debe guardar las estadísticas de colaboración de cada usuario, mediante gráficos, estadísticas, etc.
			\item La pizarra debe distinguir los distintos tipos de elementos de los agentes.
			\item La pizarra debe reconocer los tipos de archivo de los elementos.
			\item La pizarra debe ser capaz de comparar dos archivos con el objetivo de buscar diferencias.
			\item La pizarra debe permitir realizar una búsqueda de un archivo por su nombre.
			\item La pizarra debe permitir ser configurada como ``pública'' o ``privada''.
			\item El agente pedirá los credenciales del usuario al comenzar.
			\item El agente se encargará de revisar la veracidad de los credenciales del usuario.
			\item El agente adquirirá permisos y funciones dependiendo del usuario que lo ese usando.
		\end{itemize}
\item \textbf{Requisitos no funcionales}
		\begin{itemize}
			\item \textbf{De producto}
					\begin{itemize}
					\item La pizarra debe gestionar los permisos, y poder configurarlos para decidir que puede hacer cada agente.
					\item La pizarra debe bloquear a los usuarios para que no puedan acceder a problemas que no se les han asignado.
					\item La pizarra debe almacenar una contraseña y un identificador único para cada usuario y no permitir que otros accedan.
					\item La pizarra debe permitir a los agentes de mayor nivel, bloquear contenido.
					\item Por seguridad para los usuarios deben proteger su cuenta con una contraseña de más de 6 caracteres.
					\item La pizarra no revelará información personal acerca de los agentes a parte del nombre.
					\item Cada usuario accederá a la pizarra a través de un agente.
					\end{itemize}
			\item \textbf{Organizacionales}
					\begin{itemize}
					\item La plataforma se realizará con el lenguaje de programación \gloss{C++}.
					\item La plataforma estará implementada siguiendo una arquitectura de pizarra.
					\item La pizarra y los agentes siguen un modelo cliente-servidor.
					\item La comunicación entre la pizarra y el agente (modelo cliente-servidor) seguirá una encriptación SSL.
					\end{itemize}
			\item \textbf{Externos}
					\begin{itemize}
					\item La comunicación entre la pizarra y el agente (modelo cliente-servidor) será a través del protocolo HTTP.
					\item La pizarra no revelará información acerca de los agentes, excepto su nombre y número de referencia.
					\end{itemize}
		\end{itemize}
\end{itemize}

Ahora, ya tenemos unos requisitos reales de nuestra plataforma con los que poder afrontar el diseño.

\subsection{¿Son válidos estos requisitos?}

Como se trata una librería, nos surgía la duda de que si los requisitos obtenidos nos eran válidos para el diseño.\\

Pronto descubrimos que sí eran válidos, pero no eran eficientes, al estar todos mezclados era difícil saber por donde había que comenzar el diseño, por lo que optamos por organizar los requisitos de otra manera.

\section{Requisitos destinados al diseño}\label{reqdiseño}

Para que el diseño sea más conciso hemos optado por separar los requisitos en dos apartados adicionales: requisitos de la librería, que serán los requisitos de las funcionalidades que se le añaden a la pizarra y requisitos de la pizarra en sí, donde se definirá y diseñará la arquitectura propiamente dicha. 

\subsubsection{Requisitos de la pizarra}\label{reqdiseñopiz}
Lista con todos los requisitos de pizarra.
\begin{itemize}
\item \textbf{Requisitos funcionales}
	\begin{enumerate}
		\item Los agentes pueden leer, escribir y modificar la pizarra.
		\item La pizarra debe permitir gestionar los distintos tipos de agentes.
		\item Los agentes no deben poder interrelacionarse entre ellos si no es a través de la pizarra.
		\item Debe haber distintos tipos de agentes, organizados por una jerarquía de nivel (p.e. directores, profesores, alumnos).
		\item La pizarra debe permitir la creación de carpetas donde guardar los elementos para organizarlos.
		\item La pizarra debe distinguir los distintos tipos de elementos de los agentes.
		\item La pizarra debe reconocer los tipos de archivo de los elementos.
		\item La pizarra debe ser capaz de comparar dos archivos con el objetivo de buscar diferencias.
		\item La pizarra debe permitir realizar una búsqueda de un archivo por su nombre.
		\item La pizarra debe permitir ser configurada como ``pública'' o ``privada''.
	\end{enumerate}
\item \textbf{Requisitos no funcionales}
		\begin{itemize}
			\item \textbf{De producto}
	\begin{enumerate}
		\item La pizarra debe gestionar los permisos, y poder configurarlos para decidir que puede hacer cada agente.
	\end{enumerate}
	\begin{enumerate}
		\item La plataforma se realizará con el lenguaje de programación C++.
		\item La pizarra y los agentes siguen un modelo cliente-servidor.
		\item La comunicación entre la pizarra y el agente (modelo cliente-servidor) seguirá una encriptación SSL.
	\end{enumerate}
			\item \textbf{Externos}
	\begin{enumerate}
		\item La comunicación entre la pizarra y el agente (modelo cliente-servidor) será a través del protocolo HTTP.
		\item La pizarra no revelará información acerca de los agentes, excepto su nombre y número de referencia.
	\end{enumerate}
		\end{itemize}
\end{itemize}

\subsubsection{Requisitos de la librería}\label{reqdiseñolib}
Lista con todos los requisitos de librería.
\begin{itemize}
\item \textbf{Requisitos funcionales}
		\begin{itemize}
			\item La plataforma se gestiona mediante usuarios.
					\item La plataforma permitirá la creación de nuevos usuarios.
					\item La plataforma debe guardar las estadísticas de colaboración de cada usuario, mediante gráficos, porcentajes, etc.
					\item La plataforma pedirá las credenciales del usuario al comenzar.
					\item La plataforma se encargará de revisar la veracidad de los credenciales del usuario.
					\item El agente adquirirá permisos y funciones dependiendo del usuario que lo ese usando.
		\end{itemize}
\item \textbf{Requisitos no funcionales}
		\begin{itemize}
			\item \textbf{De producto}
					\begin{itemize}
					\item La plataforma debe bloquear a los usuarios para que no puedan acceder a problemas que no se les han asignado.
					\item La plataforma debe almacenar una contraseña y un identificador único para cada usuario y no permitir que otros accedan.
					\item La plataforma debe permitir a los usuarios de mayor nivel bloquear contenido.
					\item Por seguridad los usuarios deben proteger su cuenta con una contraseña de más de 6 caracteres.
					\item La plataforma no revelará información personal acerca de los usuarios a parte del nombre.
					\item Cada usuario accederá a la pizarra a través de un agente.
					\end{itemize}
			\item \textbf{Organizacionales}
					\begin{enumerate}
						\item La plataforma se realizará con el lenguaje de programación C++.
						\item La plataforma estará implementada siguiendo una arquitectura de pizarra.
					\end{enumerate}
			\item \textbf{Externos}
		\end{itemize}
\end{itemize}

Ahora sí, con estos requisitos estamos plenamente preparados para comenzar el diseño de la plataforma, eso sí, comenzando primero por la arquitectura de pizarra.