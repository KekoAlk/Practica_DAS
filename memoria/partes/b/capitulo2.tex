\chapter{Documentación}
\lettrine[lines=1,slope=4pt,findent=0pt]{U}{}na vez introducida la arquitectura de pizarra y la estructura básica de la plataforma, daremos paso a una documentación detallada de la misma. Para abordar la documentación de la plataforma ha sido necesario previamente la obtención de los requisitos, para después, una vez desarrollados y clasificados convenientemente, dar lugar a la realización de los diagramas UML correspondientes.

\section{Obtención de los requisitos}
Para abordar la obtención de los requisitos hemos partido de una posible aplicación que pudiera ser implementada con nuestra librería para, a partir de ahí, generalizar los mismos y extrapolarlos a la librería.\\

La aplicación tomada como referencia para la extracción de requisitos permite a un profesor, así como a sus alumnos subir y modificar archivos en una pizarra externa, permitiendo así la cómoda realización de trabajos en grupo por parte de los alumnos y un seguimiento por parte del profesor.\\

Pensar de un modo más aplicado nos ha permitido extraer requisitos con más facilidad, pensando en qué cosas serían necesarias para el buen funcionamiento de la aplicación.\\

Una vez extraídos los requisitos se han clasificado según lo siguiente:

\begin{itemize}
	\item \textbf{Funcionales: }Son declaraciones de los servicios que debe proporcionar el sistema. Especifica la manera en que éste debe reaccionar a determinadas entradas. Especifica cómo debe comportarse el sistema en situaciones particulares.
	\item \textbf{No funcionales: } Restricciones de los servicios o funciones ofrecidas por el sistema (fiabilidad, tiempo de respuestas, capacidad de almacenamiento, etc.). Generalmente se aplican al sistema en su totalidad. Surgen de las necesidades del usuario debido a restricciones de presupuesto, políticas de la organización, necesidad de interoperatividad, etc.
	\begin{itemize}
		\item\textbf{Del producto:} Especifican el comportamiento del producto, por ejemplo fiabilidad o rapidez.
		\item\textbf{Organizacionales:} Derivan de políticas y procedimientos existentes en la organización del cliente y del desarrollador, por ejemplo lenguajes de programación.
		\item\textbf{Requisitos externos:} Se derivan de factores externos al sistema y a su proceso de desarrollo, por ejemplo requisitos de interoperatividad o éticos.
	\end{itemize}
\end{itemize}

Para que el diseño quede más claro se han separado los requisitos en dos apartados adicionales: requisitos de la librería y requisitos de la pizarra. 

\subsection{Requisitos de la librería}
\subsubsection{Requisitos funcionales}
	\begin{enumerate}
		\item La plataforma se gestiona mediante usuarios.
		\item La plataforma permitirá la creación de nuevos usuarios.
		\item La plataforma debe guardar las estadísticas de colaboración de cada usuario, mediante gráficos, porcentajes, etc
		\item La plataforma pedirá las credenciales del usuario al comenzar.
		\item La plataforma se encargará de revisar la veracidad de los credenciales del usuario.
		\item El agente adquirirá permisos y funciones dependiendo del usuario que lo ese usando.
	\end{enumerate}
\subsubsection{Requisitos no funcionales}
\textbf{De producto}
	\begin{enumerate}
		\item La plataforma debe bloquear a los usuarios para que no puedan acceder a problemas que no se les han asignado.
		\item La plataforma debe almacenar una contraseña y un identificador único para cada usuario y no permitir que otros accedan.
		\item La plataforma debe permitir a los usuarios de mayor nivel bloquear contenido.
		\item Por seguridad los usuarios deben proteger su cuenta con una contraseña de más de 6 caracteres.
		\item La plataforma no revelará información personal acerca de los usuarios a parte del nombre.
		\item Cada usuario accederá a la pizarra a través de un agente.
	\end{enumerate}
\textbf{Organizacionales}
	\begin{enumerate}
		\item La plataforma se realizará con el lenguaje de programación C++.
		\item La plataforma estará implementada siguiendo una arquitectura de pizarra.
	\end{enumerate}

\subsection{Requisitos de la pizarra}
\subsubsection{Requisitos funcionales}
	\begin{enumerate}
		\item Los agentes pueden leer, escribir y modificar la pizarra.
		\item La pizarra debe permitir gestionar los distintos tipos de agentes.
		\item Los agentes no deben poder interrelacionarse entre ellos si no es a través de la pizarra.
		\item Debe haber distintos tipos de agentes, organizados por una jerarquía de nivel (p.e. directores, profesores, alumnos).
		\item La pizarra debe permitir la creación de carpetas donde guardar los elementos para organizarlos.
		\item La pizarra debe distinguir los distintos tipos de elementos de los agentes.
		\item La pizarra debe reconocer los tipos de archivo de los elementos.
		\item La pizarra debe ser capaz de comparar dos archivos con el objetivo de buscar diferencias.
		\item La pizarra debe permitir realizar una búsqueda de un archivo por su nombre.
		\item La pizarra debe permitir ser configurada como ``pública'' o ``privada''.
	\end{enumerate}
\subsubsection{Requisitos no funcionales}
\textbf{De producto}
	\begin{enumerate}
		\item La pizarra debe gestionar los permisos, y poder configurarlos para decidir que puede hacer cada agente.
	\end{enumerate}
\textbf{Organizacionales}
	\begin{enumerate}
		\item La plataforma se realizará con el lenguaje de programación C++.
		\item La pizarra y los agentes siguen un modelo cliente-servidor.
		\item La comunicación entre la pizarra y el agente (modelo cliente-servidor) seguirá una encriptación SSL.
	\end{enumerate}
\textbf{Externos}
	\begin{enumerate}
		\item La comunicación entre la pizarra y el agente (modelo cliente-servidor) será a través del protocolo HTTP.
		\item La pizarra no revelará información acerca de los agentes, excepto su nombre y número de referencia.
	\end{enumerate}
	
\section{Modelado UML}
Tras la extracción de los requisitos es necesario documentar la estructura de cada una de las partes que componen la plataforma a crear. Para esto usamos el lenguaje de modelado UML.\\
%UML es un lenguaje de modelado para visualizar, especificar y documentar de forma gráfica cada una de las partes del desarrollo software. \\

En primer lugar diseñaremos un diagrama de clases que permita una visión general de las clases que componen la plataforma y su relación, para después incluir un diagrama de casos de uso que permite ver de una forma más detallada los requisitos extraídos anteriormente así como su relación con los distintos actores. Para finalizar, analizaremos de forma detallada y mediante diagramas de actividad cada uno de los casos de uso resultantes.\\

\subsection{Diagramas de secuencia}

El diagrama de secuencia es un tipo de diagrama usado para modelar interacción entre objetos en un sistema según UML.

\begin{figure}[!h]
\centering
\seqIniciarSesion
\caption{Diagrama de secuencia de iniciar sesión}
\end{figure}

\subsection{Diagrama de clases}
Un diagrama de clases es un diagrama estático destinado a la programación orientada a objetos que permite describir las clases de un sistema, así como sus propiedades, operaciones, relaciones entre ellas y herencia.\\

A continuación detallamos cada una de las clases que aparecen en el diagrama centrándose en la funcionalidad y las relaciones entre ellas. Para una descripción más en profundidad de las operaciones que implementa ver Manual de uso.\\

\textbf{Pizarra:} El diagrama de clases se centra en esta clase. Contiene las operaciones necesarias para que los agentes puedan interactuar con ella. Almacena los datos relacionados con el estado de la pizarra, así como la lista de usuarios, las estadísticas, los permisos y la configuración.\\

\textbf{Agente:} Permite interactuar con la pizarra, dando opciones para leer o escribir en la misma, así como modificar su configuración o permisos. Hace las veces de interfaz al usuario para usar la pizarra. Contiene el nombre de usuario y la contraseña con la que se interactúa con la pizarra.\\

\textbf{Estado:} Contiene la lista de elementos.\\

\textbf{Elemento:} Puede ser de dos tipos, representado como herencia. Un archivo o una carpeta. Una carpeta contendrá a su vez un listado de elementos.\\

\textbf{Nivel: }Proporciona las operaciones necesarias para comprobar y editar los permisos de la pizarra.\\

\textbf{Configuración: }Permite la visualización y modificación de las configuraciones de la pizarra.\\

\textbf{Estadísticas: }Permite visualizar las estadísticas.\\

\textbf{Usuario:} Permite la modificación y visualización de los datos del usuario, como el nombre, el id, la contraseña y los permisos.


\begin{sidewaysfigure}
\centering
\clases
\caption{Diagrama de clases}
\end{sidewaysfigure}

\subsection{Diagrama de casos de uso}
Un diagrama de casos de uso permite la visualización de las actividades que se permite realizar la plataforma. A estas actividades se las denomina \textit{casos de uso}. El diagrama de casos de uso define también los actores o roles que interactúan con la aplicación y las relaciones que existen entre los distintos casos de uso. Las relaciones pueden ser de dos tipos:

\begin{itemize}
	\item \textbf{Inclusión:} Un caso de uso depende del resultado de otro.
	\item \textbf{Extensión:} Un caso de uso se extiende en otros casos que, son esencialmente similares pero varían ligeramente su comportamiento.
\end{itemize}

En nuestro caso, nuestra aplicación consta de tres actores:
\begin{itemize}
\item \textbf{Usuario}
\item \textbf{Administrador:} es un usuario normal, pero con funcionalidad añadida; sólo existe uno en el sistema.
\item \textbf{Pizarra:} es el actor principal de nuestra aplicación; es el responsable de interrelacionar a los usuario y al administrador.   
\end{itemize}

Los casos de uso más importantes para los usuarios son:
\begin{itemize}
\item \textbf{Iniciar sesión} en la pizarra
\item \textbf{Ver estadísticas} 
\item \textbf{Escribir (in)} en la pizarra
\item \textbf{Leer (out)} en la pizarra
\item \textbf{Lectura/Escritura (rd)} en la pizarra
\item \textbf{Mostrar estado}
\item \textbf{Buscar} en la pizarra
\item \textbf{Comparar} archivos en la pizarra
\item \textbf{Crear carpeta} en la pizarra
\end{itemize}

Los casos de uso para el Administrador son los mismos que los del usuario y además puede:
\begin{itemize}
\item \textbf{Crear usuario} en la pizarra
\item \textbf{Gestionar permisos} de los usuarios
\item \textbf{Configurar la pizarra}
\item \textbf{Actualizar estado} de la pizarra
\end{itemize}

Por último, los casos de uso de la pizarra son:
\begin{itemize}
\item \textbf{Mostrar estado}
\item \textbf{Buscar}
\item \textbf{Comparar} archivos
\item \textbf{Crear carpeta}
\item \textbf{Gestionar permisos} de los usuarios
\end{itemize} 

\begin{sidewaysfigure}
\centering
\casos
\caption{Diagrama de casos de uso}
\end{sidewaysfigure}

\subsection{Diagramas de actividad}
Un diagrama de actividad es una representación de un proceso de forma gráfica. Consta de una serie de símbolos que representan los distintos pasos a seguir y flechas que indican el flujo de ejecución que se sigue.\\

Como es común hemos realizado un diagrama de actividad para cada uno de los casos de uso que aparecen en el diagrama anterior.

\begin{itemize}
\item \textbf{Actualizar Pizarra:} éste es uno de los estados básicos de la pizarra, ya que cada vez se escribe o se borra algún archivo hay que hacer uso de éste.\\
Los pasos que sigue son obtener los nuevos datos de la pizarra, conectarse con la base de datos, guardar estos cambios y notificarlo.

\item \textbf{Buscar:} este caso de uso busca entre los datos de la pizarra y devuelve los datos en caso de que se hayan encontrado de acuerdo a los datos de la búsqueda.\\
Los pasos que se siguen son iniciar sesión, en caso de que los datos sean correctos, se introducen los datos de la búsqueda, se muestran estos datos si se han producido resultados, se actualiza la pizarra y finalmente se notifica.

\item \textbf{Comparar:} este caso de uso compara dos o más archivos y devuelve si se ha modificado algo y qué es lo que se ha modificado.\\
Los pasos que se siguen son iniciar sesión, se introducen los datos para comparar, se muestran los resultados si los hay, se actualiza la pizarra y se notifica.

\item \textbf{Configurar pizarra:} este caso de uso sirve para configurar la pizarra, cambiando los distintos parámetros.\\
Los pasos que siguen son iniciar sesión, se configura la pizarra, se actualiza la pizarra y se notifica al usuario.

\item \textbf{Crear carpeta:} este caso de uso permite crear nuevas carpetas a los usuarios.\\
Los pasos que sigue son se inicia sesión, se crea la carpeta, se actualiza la pizarra y se notifica al usuario.

\item \textbf{Comprobar nivel:} este caso de uso permite comprobar el nivel en el que se encuentra el usuario.\\
Los pasos a seguir son obtener los datos del usuario, conectar con la base de datos, buscar la información del usuario y mostrar las opciones del nivel.

\item \textbf{Crear usuario:} este caso de uso está restringido sólo al administrador. Crea un nuevo usuario en la pizarra.\\
Los pasos a seguir son iniciar sesión, introducir los datos del usuario, conectar con la base de datos para comprobar que los datos introducidos son correctos, se crea el nuevo usuario, se actualiza la base de datos y se notifica. En caso de que no se pueda iniciar sesión o los datos introducidos no sean válidos, también se notifica.

\item \textbf{Escribir (in):} éste es uno de las operaciones básicas de la pizarra. Escribe un nuevo dato en la pizarra.\\
Los pasos a seguir son iniciar sesión, escribir los datos que se quieren que se escriban, actualizar la pizarra y notificárselo al usuario.

\item \textbf{Lectura/Escritura (rd):} este caso de uso permite tanto leer como escribir en la pizarra.\\
Los pasos a seguir son iniciar sesión, leer datos de la pizarra, obtener la posición donde se va a escribir, escribir los datos, actualizar los datos y notificar los cambios producidos.

\item \textbf{Estadísticas:} permite ver las estadísticas de un determinado usuario.\\
Los pasos a seguir son iniciar sesión, introducir datos de las estadísticas buscadas, conectar con la base de datos, buscar las estadísticas y se muestran los resultados, si los hay.

\item \textbf{Gestionar permisos:} permite cambiar los permisos tanto de lectura como de escritura de un determinado usuario.\\
Los pasos a seguir son iniciar sesión, introducir datos del usuario a cambiar los permisos, conectar con la base de datos, se busca el usuario, si existe, se gestionan los permisos y se actualiza la base de datos, en caso contrario, se notifica que el usuario no existe.

\item  \textbf{Iniciar sesión:} se introduce el id de usuario y la contraseña para poder tener acceso.\\
Los pasos a seguir son introducir los datos (id de usuario y contraseña), conectar con la base de datos, se comprueban si los datos son válidos, si lo son se establece la conexión, se comprueban los permisos y se notifica, y si no son válidos también se notifica.

\item  \textbf{Leer (out):} leer datos de la pizarra.\\
Los pasos a seguir son iniciar sesión, leer datos que se van a leer, actualizar la pizarra y se notifica.

\item \textbf{Mostrar estado:} muestra el estado actual de la pizarra.\\
Los pasos a seguir son iniciar sesión, se muestra el estado de la pizarra, se actualiza la pizarra y se notifica al usuario.
\end{itemize}

Todos los diagramas se muestran a continuación: 

\vspace*{3cm}
\begin{figure}[!h]
\centering
\actualizarPizarra\label{fig:actualizarPizarra}
\caption{Actualizar pizarra}
\end{figure}
\newpage

\begin{figure}[!h]
\centering
\buscar\label{fig:buscar}
\caption{Buscar}
\end{figure}
\newpage

\begin{figure}[!h]
\centering
\comparar\label{fig:comparar}
\caption{Comparar}
\end{figure}
\newpage

\begin{figure}[!h]
\centering
\configurarPizarra\label{fig:configurarPizarra}
\caption{Configurar pizarra}
\end{figure}
\newpage

\begin{figure}[!h]
\centering
\crearCarpeta\label{fig:crearCarpeta}
\caption{Crear carpeta}
\end{figure}
\newpage

\begin{figure}[!h]
\centering
\comprobarNivel\label{fig:comprobarNivel}
\caption{Comprobar nivel}
\end{figure}
\newpage

\begin{figure}[!h]
\centering
\crearUsuario\label{fig:crearUsuario}
\caption{Crear usuario}
\end{figure}
\newpage

\begin{figure}[!h]
\centering
\escribir\label{fig:escribir}
\caption{Escribir}
\end{figure}
\newpage

\begin{figure}[!h]
\centering
\lecturaEscritura\label{fig:lecturaEscritura}
\caption{Lectura/Escritura}
\end{figure}
\newpage

\begin{figure}[!h]
\centering
\estadisticas\label{fig:estadisticas}
\caption{Estadísticas}
\end{figure}
\newpage

\begin{figure}[!h]
\centering
\gestionarPermisos\label{fig:gestionarPermisos}
\caption{Gestionar permisos}
\end{figure}
\newpage

\begin{figure}[!h]
\centering
\iniciarSesion\label{fig:iniciarSesion}
\caption{Iniciar sesión}
\end{figure}
\newpage

\begin{figure}[!h]
\centering
\leer\label{fig:leer}
\caption{Leer}
\end{figure}
\newpage

\begin{figure}[!h]
\centering
\mostrarEstado\label{fig:mostrarEstado}
\caption{Mostrar estado}
\end{figure}
\newpage

