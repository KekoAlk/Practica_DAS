\chapter{Documentación}
\lettrine[lines=1,slope=4pt,findent=0pt]{U}{}na vez introducida la arquitectura de pizarra y la estructura básica de la plataforma, daremos paso a una documentación detallada de la misma. Para abordar la documentación de la plataforma ha sido necesario previamente la obtención de los requisitos, para después, una vez desarrollados y clasificados convenientemente, dar lugar a la realización de los diagramas UML correspondientes.

\section{Obtención de los requisitos}
Para abordar la obtención de los requisitos hemos partido de una posible aplicación que pudiera ser implementada con nuestra librería para, a partir de ahí, generalizar los mismos y extrapolarlos a la librería.\\

La aplicación tomada como referencia para la extracción de requisitos permite a un profesor, así como a sus alumnos subir y modificar archivos en una pizarra externa, permitiendo así la cómoda realización de trabajos en grupo por parte de los alumnos y un seguimiento por parte del profesor.\\

Pensar de un modo más aplicado nos ha permitido extraer requisitos con más facilidad, pensando en qué cosas serían necesarias para el buen funcionamiento de la aplicación.\\

Una vez extraídos los requisitos se han clasificado según lo siguiente:

\begin{itemize}
	\item \textbf{Funcionales: }Son declaraciones de los servicios que debe proporcionar el sistema. Especifica la manera en que éste debe reaccionar a determinadas entradas. Especifica cómo debe comportarse el sistema en situaciones particulares.
	\item \textbf{No funcionales: } Restricciones de los servicios o funciones ofrecidas por el sistema (fiabilidad, tiempo de respuestas, capacidad de almacenamiento, etc.). Generalmente se aplican al sistema en su totalidad. Surgen de las necesidades del usuario debido a restricciones de presupuesto, políticas de la organización, necesidad de interoperatividad, etc.
	\begin{itemize}
		\item\textbf{Del producto:} Especifican el comportamiento del producto, por ejemplo fiabilidad o rapidez.
		\item\textbf{Organizacionales:} Derivan de políticas y procedimientos existentes en la organización del cliente y del desarrollador, por ejemplo lenguajes de programación.
		\item\textbf{Requisitos externos:} Se derivan de factores externos al sistema y a su proceso de desarrollo, por ejemplo requisitos de interoperatividad o éticos.
	\end{itemize}
\end{itemize}

Para que el diseño quede más claro se han separado los requisitos en dos apartados adicionales: requisitos de la librería y requisitos de la pizarra. 

\subsection{Requisitos de la librería}
\subsubsection{Requisitos funcionales}
	\begin{enumerate}
		\item La plataforma se gestiona mediante usuarios.
		\item La plataforma permitirá la creación de nuevos usuarios.
		\item La plataforma debe guardar las estadísticas de colaboración de cada usuario, mediante gráficos, porcentajes, etc
		\item La plataforma pedirá las credenciales del usuario al comenzar.
		\item La plataforma se encargará de revisar la veracidad de los credenciales del usuario.
		\item El agente adquirirá permisos y funciones dependiendo del usuario que lo ese usando.
	\end{enumerate}
\subsubsection{Requisitos no funcionales}
\textbf{De producto}
	\begin{enumerate}
		\item La plataforma debe bloquear a los usuarios para que no puedan acceder a problemas que no se les han asignado.
		\item La plataforma debe almacenar una contraseña y un identificador único para cada usuario y no permitir que otros accedan.
		\item La plataforma debe permitir a los usuarios de mayor nivel bloquear contenido.
		\item Por seguridad los usuarios deben proteger su cuenta con una contraseña de más de 6 caracteres.
		\item La plataforma no revelará información personal acerca de los usuarios a parte del nombre.
		\item Cada usuario accederá a la pizarra a través de un agente.
	\end{enumerate}
\textbf{Organizacionales}
	\begin{enumerate}
		\item La plataforma se realizará con el lenguaje de programación C++.
		\item La plataforma estará implementada siguiendo una arquitectura de pizarra.
	\end{enumerate}

\subsection{Requisitos de la pizarra}
\subsubsection{Requisitos funcionales}
	\begin{enumerate}
		\item Los agentes pueden leer, escribir y modificar la pizarra.
		\item La pizarra debe permitir gestionar los distintos tipos de agentes.
		\item Los agentes no deben poder interrelacionarse entre ellos si no es a través de la pizarra.
		\item Debe haber distintos tipos de agentes, organizados por una jerarquía de nivel (p.e. directores, profesores, alumnos).
		\item La pizarra debe permitir la creación de carpetas donde guardar los elementos para organizarlos.
		\item La pizarra debe distinguir los distintos tipos de elementos de los agentes.
		\item La pizarra debe reconocer los tipos de archivo de los elementos.
		\item La pizarra debe ser capaz de comparar dos archivos con el objetivo de buscar diferencias.
		\item La pizarra debe permitir realizar una búsqueda de un archivo por su nombre.
		\item La pizarra debe permitir ser configurada como ``pública'' o ``privada''.
	\end{enumerate}
\subsubsection{Requisitos no funcionales}
\textbf{De producto}
	\begin{enumerate}
		\item La pizarra debe gestionar los permisos, y poder configurarlos para decidir que puede hacer cada agente.
	\end{enumerate}
\textbf{Organizacionales}
	\begin{enumerate}
		\item La plataforma se realizará con el lenguaje de programación C++.
		\item La pizarra y los agentes siguen un modelo cliente-servidor.
		\item La comunicación entre la pizarra y el agente (modelo cliente-servidor) seguirá una encriptación SSL.
	\end{enumerate}
\textbf{Externos}
	\begin{enumerate}
		\item La comunicación entre la pizarra y el agente (modelo cliente-servidor) será a través del protocolo HTTP.
		\item La pizarra no revelará información acerca de los agentes, excepto su nombre y número de referencia.
	\end{enumerate}
	
\section{Modelado UML}
Tras la extracción de los requisitos es necesario documentar la estructura de cada una de las partes que componen la plataforma a crear. Para esto usamos el lenguaje de modelado UML.\\

UML es un lenguaje de modelado para visualizar, especificar y documentar de forma gráfica cada una de las partes del desarrollo software. \\

En primer lugar diseñaremos un diagrama de clases que permita una visión general de las clases que componen la plataforma y su relación, para después incluir \color{red}un diagrama de casos de uso \color{black}que permite ver de una forma más detallada los requisitos extraídos anteriormente así como su relación con los distintos actores. Para finalizar Analizaremos de forma detallada y mediante diagramas de flujo cada uno de los casos de uso resultantes.\\

\subsection{Diagrama de clases}
un diagrama de clases es un diagrama estático destinado a la programación orientada a objetos que permite describir las clases de un sistema, así como sus propiedades, operaciones, relaciones entre ellas y herencia.\\

A continuación detallamos cada una de las clases que aparecen en el diagrama centrándose en la funcionalidad y las relaciones entre ellas. Para una descripción más en profundidad de las operaciones que implementa ver Manual de uso.\\

\textbf{Pizarra:} El diagrama de clases se centra en esta clase. Contiene las operaciones necesarias para que los agentes puedan interactuar con ella. Almacena los datos relacionados con el estado de la pizarra, así como la lista de usuarios, las estadísticas, los permisos y la configuración.\\

\textbf{Agente:} Permite interactuar con la pizarra, dando opciones para leer o escribir en la misma, así como modificar su configuración o permisos. Hace las veces de interfaz al usuario para usar la pizarra. Contiene el nombre de usuario y la contraseña con la que se interactúa con la pizarra.\\

\textbf{Estado:} Contiene la lista de elementos.\\

\textbf{Elemento:} Puede ser de dos tipos, representado como herencia. Un archivo o una carpeta. Una carpeta contendrá a su vez un listado de elementos.\\

\textbf{Nivel: }Proporciona las operaciones necesarias para comprobar y editar los permisos de la pizarra.\\

\textbf{Configuración: }Permite la visualización y modificación de las configuraciones de la pizarra.\\

\textbf{Estadísticas: }Permite visualizar las estadísticas.\\

\textbf{Usuario:} Permite la modificación y visualización de los datos del usuario, como el nombre, el id, la contraseña y los permisos.


{\color{red}{\huge PROVISIONAL}}

\begin{landscape}
\begin{figure}[!h]
\centering
\clases
\caption{Diagrama de clases}
\end{figure}
\end{landscape}

\subsection{Diagrama de casos de uso}

\begin{figure}[!h]
\centering
\actualizarPizarra
\caption{Actualizar pizarra}
\end{figure}
\newpage

\begin{figure}[!h]
\centering
\buscar
\caption{Buscar}
\end{figure}
\newpage

\begin{figure}[!h]
\centering
\comparar
\caption{Comparar}
\end{figure}
\newpage

\begin{figure}[!h]
\centering
\configurarPizarra
\caption{Configurar pizarra}
\end{figure}
\newpage

\begin{figure}[!h]
\centering
\crearCarpeta
\caption{Crear carpeta}
\end{figure}
\newpage

\begin{figure}[!h]
\centering
\comprobarNivel
\caption{Comprobar nivel}
\end{figure}
\newpage

\begin{figure}[!h]
\centering
\comprobarNivel
\caption{Comprobar nivel}
\end{figure}
\newpage

\begin{figure}[!h]
\centering
\crearUsuario
\caption{Crear usuario}
\end{figure}
\newpage

\begin{figure}[!h]
\centering
\escribir
\caption{Escribir}
\end{figure}
\newpage

\begin{figure}[!h]
\centering
\lecturaEscritura
\caption{Lectura/Escritura}
\end{figure}
\newpage

\begin{figure}[!h]
\centering
\estadisticas
\caption{Estadísticas}
\end{figure}
\newpage

\begin{figure}[!h]
\centering
\gestionarPermisos
\caption{Gestionar permisos}
\end{figure}
\newpage

\begin{figure}[!h]
\centering
\iniciarSesion
\caption{Iniciar sesión}
\end{figure}
\newpage

\begin{figure}[!h]
\centering
\leer
\caption{Leer}
\end{figure}
\newpage

\begin{figure}[!h]
\centering
\mostrarEstado
\caption{Mostrar estado}
\end{figure}
\newpage

