\section*{Objetivos y motivaciones}
Tanto en el ámbito informático académico como en el profesional, en ocasiones puede ser necesario un cambio de sistema operativo a la hora de trabajar. Esto puede estar ocasionado tanto por requisitos del lenguaje a utilizar, por una arquitectura más adecuada o simplemente por cuestión de preferencias. Pero estos cambios implicarían una CPU para cada sistema operativo, lo que lo haría terriblemente ineficiente desde el punto de vista económico, ergonómico y logístico. \\
Por ello es necesario conocer la arquitectura del software que permita escoger entre más de un sistema operativo, y que el adicional (aún sin llegar al nivel de eficiencia del sistema nativo) trabaje a un nivel óptimo.\\
Pero además de observar y comprender la arquitectura de un software conocido y utilizado comercialmente, también son necesarios los conocimientos necesarios para poder diseñar a un nivel de arquitectura y diseño el funcionamiento de una plataforma. Para ello crearemos una biblioteca en la que cualquier usuario podrá programar un agente capaz de leer y escribir en una pizarra potencialmente remota.\\
Por tanto los dos principales objetivos de la práctica serán:
\begin{itemize}
\item Definir la arquitectura de un sistema de virtualización.
\item Describir la arquitectura y diseño de una plataforma que implemente la arquitectura de pizarra que pueda ser utilizada por otros programas.
\end{itemize}

\section*{Estructura de la memoria}
Debido a las dos partes claramente diferenciadas e la práctica, se harán dos secciones claramente diferenciadas :
\begin{itemize}
\item Parte de OBSERVACIÓN: en la que tras una pequeña introducción se detallará la arquitectura del sistema escogido
\item Parte de DISEÑO: se empezará describiendo qué es una arquitectura de pizarra, qué componentes la forman, etc para posteriormente analizar los requisitos y los diagramas UML de los que constará nuestra aplicación, para finalmente, dar un manual básico con las operaciones básicas de nuestra aplicación. 
\end{itemize}

\section*{¿Por qué GitHub?}
Para la realización de esta práctica hemos utilizado el software \textbf{GitHub}, propiedad de 	{GitHub Inc}, el cual nos ha permitido trabajar sobre un repositorio al que todos los miembros del grupo teníamos acceso. Hemos elegido usar GitHub por los siguientes motivos:
\begin{itemize}
	\item Es un conocido programa para desarrolladores de software, lo que nos ha permitido trabajar de una forma profesional.
	\item Al ser un alto número los integrantes del grupo, nos ha sido más fácil y sencillo trabajar mediante repositorios. 
	\item Pero principalmente debido a que dicho programa sigue una arquitectura de pizarra y soporta el conocido framework ``Ruby on Rails", lo que nos acercaba a un entorno práctico a los dos aspectos a tratar en la práctica: los framework y la arquitectura de pizarra.
\end{itemize}

Todos los archivos de esta práctica se encontrarán disponibles en la dirección
\url{https://github.com/KekoAlk/Practica_DAS}.

\section*{¿Por qué \LaTeX?}
\LaTeX~ es un sistema de composición de textos profesional que permite separar de forma fácil un único archivo en varios, de forma que trabajar de forma distribuida sea bastante cómoda. Además, los archivos son texto plano por lo que, junto a GitHub, hace que se pueda ver qué se ha añadido o borrado en un documento de forma sencilla. También es un sistema que separa el contenido de la presentación, por lo que facilita centrarse en el contenido y no tanto en su presentación.\\




