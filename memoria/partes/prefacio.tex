\lettrine[lines=1,slope=4pt,findent=0pt]{E}{}sto es un prefacio.\\

\color{red}
FALTA:
\begin{itemize}
\item Explicar los objetivos y motivaciones de la práctica
\item Recalcar que hay dos partes y en qué se centra cada una.
%\item Explicar la forma de trabajo \textit{(De forma distribuida mediante GitHub)}
\end{itemize}
\color{black}

Para la realización de esta práctica hemos utilizado el software {\bf GitHub}, propiedad de 	GitHub Inc, el cual nos ha permitido trabajar sobre un repositorio al que todos los miembros del grupo teníamos acceso. Hemos elegido usar GitHub por los siguientes motivos:
\begin{itemize}
	\item Es un conocido programa para desarrolladores de software, lo que nos ha permitido trabajar de una forma profesional
	\item Al ser un alto número los integrantes del grupo, nos ha sido más fácil y sencillo trabajar mediante repositorios, 
	\item Pero principalmente debido a que dicho programa sigue una arquitectura de pizarra y soporta el conocido framework 'Ruby on Rails', lo que nos acercaba a un entorno práctico a los dos aspectos a tratar en la práctica: los framework y la arquitectura de pizarra.
\end{itemize}

\color{red}
FALTA:
\begin{itemize}
%\item Explicar la forma de trabajo \textit{(De forma distribuida mediante GitHub)}
\item Si se hace código, recalcarlo mucho.
\item Dar paso al inicio de la primera parte.
\end{itemize}
\color{black}