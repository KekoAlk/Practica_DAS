\chapter[Descripción de la arquitectura]{Descripción de la arquitectura de virtualización}

\lettrine[lines=1,slope=4pt,findent=0pt]{U}{}na vez introducido el vocabulario básico y tras haber indagado un poco más en la materia vamos a centrarnos en la arquitectura como tal.\\

\section{Tipos de Virtualización}
En el apartado anterior ya se pudo vislumbrar que existen diferentes formas de virtualización, y por ello, vamos a hacer un pequeño análisis de cada uno, pero antes nos interesa conocer el término \emph{\gloss{HYP}}, ya que es el elemento central de un sistema de máquinas virtuales.

\subsection{¿Qué es un hipervisor?}
El \emph{\gloss{HYP}} o \emph{\gloss[long]{VMM}} se trata de una plataforma que permite aplicar diversas técnicas de control para utilizar, al mismo tiempo, diferentes sistemas operativos en una misma computadora.\\

Se trata de un elemento software que dependiendo de cómo se sitúe en relación con el Hardware da lugar a dos maneras diferentes de virtualizar, dos tipos de \emph{\gloss{HYP}}\cite{tipoship}:

\subsection{Hipervisor de Tipo 1 o \emph{Nativo}}
 El software del hipervisor se ubica directamente entre el hardware y las distintas máquinas virtuales, para ofrecer la funcionalidad descrita, siguiendo la siguiente estructuración:

\begin{figure}[H]
\begin{center}
\figura{1}
\end{center}
\caption[Hipervisor Tipo 1]{Esquema de un hipervisor de primer nivel}
\end{figure}

Este tipo de \emph{hipervisor} también es conocido como \emph{unhosted} o \emph{bare metal}, que en inglés hacen referencia a que no es huésped o que se ejecuta a bajo nivel, respectivamente.\\

Dentro de este tipo se encuentran VMware ESXi, VMware ESX y Microsoft Hyper-V Server, pero nos gustaría presta una atención especial a \gloss{XEN} por ser un hipervisor de código abierto desarrollado por la Universidad de Cambridge\cite{proyectoxen}\cite{proyectoxen2}.
\subsection{Hipervisor de Tipo 2 o \emph{Huésped}}
Es una arquitectura alternativa para la máquina virtual insertando una capa de virtualización encima del sistema operativo \emph{host} o huésped, siendo éste responsable de administrar el hardware. Los sistemas operativos invitados se instalarán encima del nivel de virtualización, en máquinas virtuales. Tiene la siguiente estructura:

\begin{figure}[H]
\begin{center}
\figura{2}
\end{center}
\caption[Hipervisor Tipo 2]{Esquema de un hipervisor de segundo nivel}
\end{figure}

Este tipo de hipervisor tiene una ventaja muy destacada, el usuario puede instalar esta arquitectura de máquina virtual sin modificar el sistema operativo host pudiendo descansar en el sistema operativo host para entregar los controladores de dispositivos y otros servicios de bajo nivel (se simplifica el diseño de la máquina virtual y facilita la implementación).\\

Algunos de los hipervisores tipo 2 más utilizados son: Oracle: VirtualBox, VirtualBox OSE, VMware: Workstation; siendo éste último en el que más nos vamos a centrar.

\section{Componentes}

\textcolor{red}{PEQUEÑO RESUMEN PARA INTRODUCIR QUE SE VA A HACER UN DESGLOSE DE CADA PARTE.}

\subsection{Máquina Virtual}

\textcolor{red}{AQUI VA EL ESQUEMA GENERAL DE UN VM.}


\subsection{Hipervisor}

\textcolor{red}{AQUI SE DESCRIBEN LOS COMPONENTES DE UN HIPERVISOR GENERAL.}

\section{Esquema general}

\textcolor{red}{AQUI VA EL ESQUEMA CON TODO MEZCLADO PARA QUE SE VEA A NIVEL GENERAL TODO.}
