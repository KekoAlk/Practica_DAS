\chapter{Introducción}
\lettrine[lines=1,slope=4pt,findent=0pt]{E}{}n esta primera parte de la memoria nos vamos a encargar de hacer una descripción a nivel arquitectónico de los sistemas de virtualización de las aulas de la Universidad Rey Juan Carlos.\\

Para llevar a cabo esta tarea hemos decidido que la mejor opción pasa por describir o analizar qué es un sistema de virtualización, para después poder centrarnos en el caso concreto de las aulas de la URJC, que se encuentran virtualizadas con VMware\cite{vmware} lo que significa que en el caso concreto nos centraremos en VMware, eso sí, haciendo continuas analogías a lo que se ve en las aulas.\\

Pero para entrar un poco en materia vamos a intentar explicar brevemente qué es esto de la \emph{virtualización}.

\section{¿Qué es la virtualización?}
Con la intención de hacernos una idea precisa del término \emph{virtualización} hemos buscado diversas definiciones, de las cuales nos vamos a quedar con:
\begin{quote}
\emph{<<Virtualización es la creación -a través de software- de una versión virtual de algún recurso tecnológico, como puede ser una plataforma de hardware, un sistema operativo, un dispositivo de almacenamiento u otros recursos de red.>>}
\begin{flushright}
--Wikipedia\cite{defvirwiki}
\end{flushright}
\end{quote}

Es otras palabras, se trata de la creación mediante software de elementos hardware virtuales. Un buen ejemplo de esto es cuando particionamos un disco duro; físicamente tenemos un único \emph{\gloss[short]{HDD}} pero a nivel de software existen dos, pues el disco duro esta \emph{virtualizado} en dos particiones, el sistema operativo se encarga de abstraer las propiedades físicas del disco.\\

Dando una vuelta de tuerca más, podemos definir \emph{virtualización} como el proceso por el cual una capa de software \emph{\gloss[long]{VMM}} abstrae los recursos de la computadora física al Sistema Operativo o \emph{\gloss{MAQVIR}}.\\

Cabe destacar el uso de los términos \emph{\gloss{MAQVIR}} y \gloss[short]{VMM}, pues son el eje central de los sistemas de virtualización y conviene introducirlos. Y es por esto que nuestro siguiente apartado está dedicado a ellas.

\section{¿Qué es una máquina virtual?}

Al igual que antes, encontramos diversas acepciones de \emph{\gloss{MAQVIR}}, pero vamos a partir de la definición de \emph{Wikipedia} pues es la que consideramos más completa.

\begin{quote}
\emph{<<Una máquina virtual es un software que simula a una computadora y puede ejecutar programas como si fuese una computadora real. Este software en un principio fue definido como \textquotedblleft un duplicado eficiente y aislado de una máquina física\textquotedblright.>>}
\begin{flushright}
--Wikipedia\cite{defmaqvirwiki}
\end{flushright}
\end{quote}

O lo que es lo mismo, un programa o aplicación software que simula de manera total o parcial una computadora de forma que el usuario final no note la diferencia.\\

Podemos encontrar dos tipos de éstas dependiendo de cómo se ejecuten en el sistema:

\begin{itemize}
\item \textbf{Máquinas virtuales de proceso} o \gloss{MAQVIR} de aplicación, se ejecuta como un proceso normal dentro de un sistema operativo y soporta un solo proceso. Su objetivo es el de proporcionar un entorno de ejecución independiente del sistema operativo y del hardware. Una \gloss{MAQVIR} de proceso muy popular es la de Java o \emph{\gloss[long]{JVM}}.
\item \textbf{Máquinas virtuales de sistema} o máquinas virtuales de hardware, que permiten a la máquina física multiplicarse entre varias máquinas virtuales, cada una con su propio sistema operativo. A la capa de software que se permite la virtualización se le llama \emph{monitor de máquina virtual} o \gloss[short]{VMM}, anteriormente mencionado.
\end{itemize}

Como es obvio nosotros nos vamos a centrar en el último tipo, puesto que vamos a analizar la virtualización en las computadoras de la universidad, y éstas se ejecutan como un sistema operativo propio y no como una aplicación de éste.

\section{Condiciones para la virtualización}

A la hora de crear un sistema virtualizado existen propiedades que hacen que se pueda evaluar, es decir \emph{¿Cuán bueno es el sistema?}, estas propiedades se refieren al el monitor de la máquina virtual o \emph{VMM}. Listémoslas pues:
\begin{itemize}
\item \textbf{Equivalencia o fidelidad:} Un programa corriendo bajo el VMM debe exhibir un comportamiento  esencialmente idéntico a aquel demostrado cuando se ejecuta directamente en una máquina equivalente.
\item \textbf{Control de recursos y seguridad:} El \gloss[short]{VMM} debe estar en completo control de los recursos virtualizados.
\item \textbf{Eficiencia:} La mayor parte de las instrucciones de máquina debe ser ejecutada sin la intervención del \gloss[short]{VMM}.
\end{itemize}

Para llevar a cabo una virtualización del sistema que cumpla estas propiedades, Popek y Goldberg escribieron en un artículo en 1974 en la revista \emph{Communications of the \gloss[short]{ACM}}\cite{reqvir}, qué condiciones se han de dar para una virtualización eficiente, en dicho artículo realizan un análisis del repertorio de instrucciones del computador o \emph{\gloss[long]{ISA}}.\\

Dividiendo las instrucciones en:
\begin{itemize}
\item \textbf{Instrucciones privilegiadas:} Las instrucciones que sólo funcionan en modo kernel del procesador y no lo hacen en modo usuario.
\item \textbf{Instrucciones sensibles de control:} Las instrucciones que al ejecutarse cambian la configuración del sistema de alguna manera.
\item \textbf{Instrucciones sensibles de comportamiento:} Aquellas que dependen del estado de los recursos del sistema en el momento que se ejecutan. 
\end{itemize}

Y como resultado de este análisis formularon los siguientes teoremas.
\begin{teorema}
Para cualquier computadora convencional de tercera generación, se puede construir un \gloss[short]{VMM} efectivo si el conjunto de instrucciones sensibles es un subconjunto de las instrucciones privilegiadas.
\end{teorema}

Este teorema implica que para construir un \gloss[short]{VMM} basta con que todas las instrucciones que pueden afectar a su correcto funcionamiento, las instrucciones sensibles, pasen siempre el control al \gloss[short]{VMM}. Esto garantiza la propiedad de control de recursos y seguridad. En cambio, las instrucciones no privilegiadas se deberán ejecutarse de manera nativa, para lograr eficiencia.

\begin{teorema}
Una máquina convencional de tercera generación es recursivamente virtualizable si es virtualizable y se puede construir para ella un \gloss[short]{VMM} sin ninguna dependencia de sincronización.
\end{teorema}

Este teorema es fundamental, en lo que a nosotros se refiere, pues este último teorema no se cumple para arquitecturas de computadores x86, que casualmente son las que utilizan los computadores de la universidad.\\

Ahora centrémonos en la arquitectura de los sistemas de virtualización.