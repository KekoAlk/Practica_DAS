\chapter{VMware: Workstation}
\lettrine[lines=1,slope=4pt,findent=0pt]{E}{}n este capítulo nos vamos a centrar en la plataforma de VMware, VMware: Workstation\cite{vmwarework}, los pasos a seguir serán: analizar y ver qué tipo de virtualización utiliza, describir sus principales características y dar ejemplos sobre su uso e instalación.\\

Pero, antes de nada, es necesario explicar por qué escogemos esta plataforma.

\section{¿Por qué VMware: Workstation?}
Aunque la respuesta es obvia era necesario dejar una constancia explícita de ello: hemos escogido esta plataforma pues es la que se encuentran normalmente instalada en los ordenadores de las aulas y laboratorios de la Universidad Rey Juan Carlos.\\

Para llegar a esta conclusión basta con tener acceso de uno de estos ordenadores de sobremesa y los encendamos para su uso, al arrancar descubriremos que sale una ventana con el logo de VMware y con título \emph{VMware: Workstation vX.Y build ZZZZ} siendo X, Y y Z los valores de la versión y compilación utilizada en ese ordenador.

\section{Arquitectura VMware: Workstation}
Vamos pues, a analizar detalladamente la arquitectura de esta plataforma en concreto, para ello vamos a fijarnos en sus especificaciones, en el tipo de virtualización y consultaremos la pagina oficial\cite{vmwarework}\cite{vmwareworkhelp}.

\subsection{Tipo de virtualización}
Para esta parte, al igual que para la mayoría, podemos saberlo simplemente mediante la observación y con ayuda de los conocimientos previos arriba mencionados (\textit{Véase apartados \ref{tiposvir}, \ref{tiposvir1} y \ref{tiposvir2}}).\\

Fijándonos en el proceso de arranque de los ordenadores de la URJC es fácil deducir que nos encontramos ante un tipo 2 de hipervisor, ya que para que arranque la máquina virtual y VMware, primero es necesario que arranque el ordenador con \emph{$Windows^{\textregistered}$ XP}.\\

Si buscamos información en la web\cite{refvmware1}, y especialmente en la página oficial\cite{vmwareworkguiaso}\cite{refvmware2}, son muchas las referencias que se hacen a un \emph{SO host}, elemento que sólo se incluye en el tipo 2 de virtualización, siendo este su elemento característico.\\

Por lo tanto concluimos que VMware: Workstation (y por lo tanto los ordenadores de la URJC también) necesitan de un Sistema Operativo base que cumpla los requisitos mínimos. Se trata por tanto de una virtualización de segundo nivel.

\subsection{Especificaciones técnicas}
Ahora vamos a comprobar las distintas especificaciones que tiene esta plataforma en particular, qué la difiere de un sistema genérico de virtualización de segundo nivel.\\

Entre las que hemos encontrado nos gustaría destacar:

\begin{itemize}
\item Es la herramienta con mayor soporte en cuanto a empresas se refiere del mercado.
\item Especialmente diseñada para dar funcionalidad a múltiples computadoras.
\item Soporta redes virtuales, lo que permite un mayor control sobre las conexiones que se realizan.
\item Soporta la virtualización de sistemas \gloss{x64} en ordenadores \gloss{x86}.
\end{itemize}

Pero sin duda si queremos explicar por qué la universidad escogió este sistema, se debe a su potencia para un gran número de ordenadores; tal y como se da el caso de la universidad, donde prácticamente todos los computadores se encuentran virtualizados.

\subsection{Esquema de arquitectura}
Por tanto, siguiendo la línea del capítulo anterior, este sería el esquema de la arquitectura instalada en los ordenadores de las aulas de la universidad:

\begin{figure}[H]
\begin{center}
\figura{vmware}
\end{center}
\caption[Arquitectura VMware: Workstation]{Esquema de la arquitectura de los sistemas virtualizados implementada en la universidad. VMware: Workstation}
\end{figure}

\section{Instalación y uso}
En esta sección nos gustaría explicar brevemente cómo creemos que se instaló dicho software en los ordenadores de la universidad.\\

Por comenzar por algún lado, empecemos analizando el sistema operativo huésped\cite{vmwareinst}\cite{vmwareinst2}.\\

\subsection{Instalación \emph{fresh} de Windows$^{\textregistered}$}
Indagando por la web, hemos encontrado en diversos \emph{post}, \emph{guías}, \emph{tutoriales} y \emph{blogs} una continúa mención a una \emph{instalación fresh $Windows^{\textregistered}$ XP}, bastante relacionada con lo que a nosotros nos incumbe, los ordenadores de la URJC.

Esta \emph{instalación} no es otra cosa que un sistema optimizado para ser el huésped de un sistema de virtualización de segundo nivel.\\

Esto es exactamente lo que encontramos en los ordenadores de la universidad; cuando arrancamos los ordenadores, en el arranque, nos da a escoger entre varias opciones y generalmente, la que escogemos para darles su uso normal es \emph{\textquotedblleft windows xp vmware worstation\textquotedblright}, donde, al escoger dicha opción, nos arranca un entorno windows minimalista (no posee ninguna aplicación externa) a excepción de la aplicación de VMWare.

\subsection{Funcionamiento básico}

El funcionamiento de VMware es simple, vamos a detallar el proceso que sigue un computador de la universidad desde que se pulsa el botón de arranque hasta que se inicia la máquina virtual:

\begin{enumerate}
\item Se arranca el ordenador.
\item En la selección de arranque, escogemos iniciar con VMware.
\item Se inicializa el sistema operativo huésped.
\item Se inicia automáticamente la aplicación de VMware.
\item En la aplicación se introducen los datos de la máquina virtual (para saber cuál escoger entre todas las disponibles en el servidor).
\item El servidor le devuelve la información a la aplicación.
\item Se inicia la maquina virtual.
\item Se pide el usuario y contraseña, cómo un ordenador normal, pero en este caso serán los valores que se usan para todas las cuentas de la URJC.
\item El ordenador esta listo para su uso.
\end{enumerate}

Aunque parece un proceso complicado, en realidad es de lo más simple y conciso. Esto permite distribuir los permisos en los ordenadores y evitar que se instalen programas innecesarios, al tratarse de una máquina virtual, si se requiere la instalación de una aplicación, con instalarse una vez bastaría y ésta se encontraría en todos los ordenadores de la universidad.
