% Aquí van algunos comandos para no hacer demasiado largo el preámbulo.


% Separa la lista de figuras por partes
% % % % % % % % % % % % % % % % % % % % % % % % % % % % % % % % % % % % % % % %
%
% NOTA IMPORTANTE: Este comando SIEMPRE debe ir antes de la inclusión del paquete hyperref.
%
% % % % % % % % % % % % % % % % % % % % % % % % % % % % % % % % % % % % % % % % 
% Adaptado de:
%http://tex.stackexchange.com/questions/52746/include-chapters-in-list-of-figures-with-titletoc
\makeatletter
\def\thisparttitle{}\def\thispartnumber{}
\def\thischaprertitle{}\def\thischapternumber{}
\newtoggle{noFigs}
\newtoggle{noFig}

\apptocmd{\@part}%
  {\gdef\thisparttitle{#1}\gdef\thispartnumber{\thepart}%
    \global\toggletrue{noFig}}{}{}
    
\apptocmd{\@chapter}%
  {\gdef\thischaprertitle{#1}\gdef\thischapternumber{\thechapter}
  	\global\toggletrue{noFigs}}{}{}

% the figure environment does the job: the first time it is used after a \chapter command, 
% it writes the information of the chapter to the LoF
\AtBeginDocument{%
  \AtBeginEnvironment{figure}{%
    \iftoggle{noFigs}{
    	\iftoggle{noFig}{
		    \addtocontents{lof}{\protect\contentsline {chapter}%
	              {\protect\numberline {\thispartnumber} 
	              {\thisparttitle}}{}{} }
		    \global\togglefalse{noFig}
		}{}
		    \addtocontents{lof}{\protect\contentsline {chapter}%
		            {\protect\numberline {\thischapternumber} {\thischaprertitle}}{}{} }
		    \global\togglefalse{noFigs}
	}{}
  }%
}

\makeatother
